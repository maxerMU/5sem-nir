\chapter{Анализ предметной области}

В данной работе рассматриваются подходы к построению высоко нагруженных данными систем. Это системы, которые могут принимать,
обрабатывать и генерировать большие объемы данных
разной природы и из разных источников с течением
времени, организованные с помощью различных
технологий и с целью извлечения ценности из данных
для различных типов предприятий \cite{cyberlenin1}.

C такими системами связан термин хайлоад. Это нагрузка, с которой не справляется аппаратное обеспечение из-за различных технических ограничений, к примеру, из-за нехватки процессорного времени или памяти \cite{bunin1}.

Высоко нагруженные системы должны обладать рядом свойств. Из них наибольшее значение в большинстве программных систем имеют следующие \cite{kleppman}:
\begin{itemize}
	\item надежность;
	\item масштабируемость;
	\item удобство сопровождения.
\end{itemize}

Надежность определяет устойчивость работы системы как при благоприятных условиях, так и при сбоях. В данном случае сбоем называется отклонение от штатной работы одного или нескольких компонентов системы, но всей целиком \cite{kleppman}.

Под масштабируемостью понимают способность системы, процесса или сети расти и справляться с возросшим спросом. Существует два типа масштабирования \cite{cyberlenin1}:
\begin{itemize}
	\item вертикальное, при таком подходе серверную часть приложения просто переводят на более мощную машину;
	\item горизонтальное, в этом случаем происходит перераспределение нагрузки с одной машины на несколько. 
\end{itemize}

Сопровождением системы называется исправление ошибок,
поддержание работоспособности его подсистем,
расследование отказов, адаптацию к новым платформам,
модификацию под новые сценарии использования и
добавление новых возможностей \cite{cyberlenin1}. Отсюда следует, что система должна быть способна к добавлению нового функционала или изменению сценариев поведения без внесения изменений в существующий программный код.

Не существует единственного правильного решения для построения высоко нагруженных данными систем, так как каждый подход, выбираемый при проектировании системы, должен быть связан с входными данными и конкретными задачами, решаемыми программным обеспечением. Существуют различные паттерны проектирования таких систем, они будут рассмотрены в следующей главе.